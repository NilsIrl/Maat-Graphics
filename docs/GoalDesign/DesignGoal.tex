\documentclass[12pt,a4paper]{article}
\usepackage{float}
\usepackage{listings}

\usepackage{lscape}
\usepackage{tabularx}
\usepackage{graphicx}
\usepackage{lmodern}
\usepackage{fancyhdr}
\usepackage[font={small,it}]{caption}
\usepackage{amssymb}
\usepackage{amsmath}
\usepackage{listings}
\usepackage{color}
\usepackage{tikz}

\definecolor{dkgreen}{rgb}{0,0.6,0}
\definecolor{gray}{rgb}{0.5,0.5,0.5}
\definecolor{mauve}{rgb}{0.58,0,0.82}

\lstset{frame=tb,
  language=Java,
  aboveskip=3mm,
  belowskip=3mm,
  showstringspaces=false,
  columns=flexible,
  basicstyle={\small\ttfamily},
  numbers=none,
  numberstyle=\tiny\color{gray},
  keywordstyle=\color{blue},
  commentstyle=\color{dkgreen},
  stringstyle=\color{mauve},
  breaklines=true,
  breakatwhitespace=true,
  tabsize=3
}

    \makeatletter
    \renewcommand*\env@matrix[1][*\c@MaxMatrixCols c]{%
       \hskip -\arraycolsep
       \let\@ifnextchar\new@ifnextchar
       \array{#1}}
    \makeatother

% You can do italics, bold and other formatting techniques by using functions such as \textbf{} or \small
\title{\textbf{Maat Graphics\\ Design Goal }\\ \large }
\author{Lilith Wynter}
\date{7th September, 2018}


% Headers and footers
\rhead{}
\lhead{Lilith Wynter}
\rfoot{Page \thepage}

\captionsetup{justification=raggedright,singlelinecheck=false}

%%%%%%%%%%%%%%%%%%%%% Begin the Actual Document %%%%%%%%%%%%%%%%%%%%%
\begin{document}

% Creates the title and author, date etc that you set up before
\maketitle

% This sets the first page style as nothing for a few reasons including removing the page number and headers and footers
\thispagestyle{empty}


%%%
% Uncomment for future projects where you want a table of contents and proper report style numbering for that TOC

% \newpage
% \pagenumbering{roman}
% \tableofcontents

% \newpage
% \listoffigures

%%%

\pagenumbering{arabic} % Hindu arabic numbering is the English numbering system, start numbering pages from this point on

\newpage
\tableofcontents

\newpage
\section{Introduction}
\begin{itemize}
\item "User" refers to the programer using the Maat-Graphics.
\end{itemize}
\section{Graphics API}
\subsection{Window and Context Handling}
\subsubsection{Window.rs}
The purpose of this is to handle everything to do with the window, holds both Vulkan and OpenGL versions. It is one of the lowest files in the program.
\\
This shouldn't only update with Winit and Glutin crate updates and when currently unused window functions are needed in a higher context.
\begin{itemize}
\item Handles window creation.
\item Handles window resize.
\item Gets drawing context.
\item Gets the event loop.
\item Show/Hide Mouse.
\end{itemize}
\textbf{Dependencies}
\begin{itemize}
\item None
\end{itemize}
\subsubsection{Graphics.rs - CoreRender}
The purpose of this is to simply hold the trait that is shared between openGL and Vulkan that the User has access to all of the trait functions and only difference for the User will be calling the new function from either openGL or Vulkan, where the new functions are defined as the impl of the respective structs.\\
Elaborated In ResourceManager and respective Graphics API sections.
\\
This will rarely be updated unless there is a fundemental change or a new function is needed to be exposed to the User.
\subsection{Vulkan}
\subsubsection{VkMaat.rs}
The purpose of this file is to impl the trait CoreRender from Graphics.rs. the veins of Maat-Graphics.\\
\begin{itemize}
\item Creates VkWindow from Window.rs, passing parameters from Settings.rs
\item Creates TextureShader, ForwardShader and FinalShader using the new() from their respective files.
\end{itemize}
\textbf{Functions}
\begin{itemize}
\item clear$\_$screen() \\Does as it says, clears the current framebuffer.
\item pre$\_$draw() \\This is used to do anything that needs to be completed before drawing is attempted. In this case recreating the swapchain when nessarily, usually only happens after window creation and screen resizes.
\item draw() \\Takes in all the draw$\_$calls, Frame buffers are composited in this order, TextureShader $->$ ForwardShader $->$ FinalShader. \\Both TextureShader and Forward shader output a single image each, they need to be passed to the FinalShader for drawing.
\item post$\_$draw() \\Not used here, but is used for anything that needs to be disabled or unbounded after the entire draw.
\item swap$\_$buffers()\\Not used here, is used for openGL to swap current drawing image.
\item screen$\_$resized()\\This function should be called everytime the screen is resized and should put in motion nessesary functions for handling screen resizes.
\end{itemize}
\textbf{Dependencies}
\begin{itemize}
\item VkWindow
\item ResourceManager
\item TextureShader
\item ForwardShader
\item FinalShader
\item Camera
\end{itemize}
\subsubsection{resource.rs}
The purpose of this file is to load and handle texture and model resources, most of these functions will be called directly from whatever impl CoreRender.\\
\begin{itemize}
\item Creates a threadpool that will be used to load resources in.
\item Holds Vec of LoadableObjects which contains location of file and the reference name, as well as wether or not it has been loaded in or not.
\end{itemize}
\textbf{Functions}
\begin{itemize}
\item new() \\Creates empty Vec of loadableObjects and a ThreadPool
\item TODO: add$\_$texture()\\Will add a loadableObject to the Vec without loading it into memory.
\item get$\_$texture\\Returns Texture image for use elsewhere if it is loaded in
\item load$\_$texture() \\Initates a new LoadableObject and creates a new thread to load the new textures into memory.
\item load$\_$texture$\_$into$\_$memory() \\Loads textures into immutableImage format onto the GPU, usually called from a thread.
\end{itemize}
\textbf{Dependencies}
\begin{itemize}
\item None
\end{itemize}
\subsubsection{finalshader.rs}
The purpose of this file is to hold the FinalShader struct used in VkMaat.\\
\begin{itemize}
\item Deals with the final frame buffer to draw "directly" to the screen.
\item Handles passing uniform variables from cpu to shader, data comes from parameters.
\item Loads in the Shader and any image attachments related to this specific framebuffer. (excluding attachments from other Framebuffers).
\item Loads in the Vertices and Indices for drawing the single 2D quad used in this framebuffer.
\end{itemize}
\textbf{Functions}
\begin{itemize}
\item create()\\ This function is to inits all the data the TextureShader holds, including the framebuffer, uniform buffer, vertex and indices, renderpass (Attachments), and Pipeline (Shader).
\item empty$\_$framebuffer()\\ discards all avalible images for drawing, thus triggering recreate$\_$framebuffer.
\item recreate$\_$framebuffer() \\Creates fresh drawing contexts.
\item begin$\_$renderpass() \\Sets the frame buffer to be drawn to
\item draw() \\Takes in Both images from TextureShader and ForwardShader and passes them to the shader along with uniform variables and draws quad over the entire screen.
\item end$\_$renderpass()\\Finishes the current framebuffer
\end{itemize}
\textbf{Dependencies}
\begin{itemize}
\item VkFinal.vert
\item VkFinal.frag
\end{itemize}
\subsection{OpenGl}
\subsubsection{GlMaat.rs - (Currently rawgl.rs needs renaming)}
The purpose of this file is to impl the trait CoreRender from Graphics.rs. the veins of Maat-Graphics.\\
\end{document}